%%%%%%%%%%%%%%%%%%%%%%%%%%%%%%%%%%%%%%%%%
% Beamer Presentation
% LaTeX Template
% Version 1.0 (10/11/12)
%
% This template has been downloaded from:
% http://www.LaTeXTemplates.com
%
% License:
% CC BY-NC-SA 3.0 (http://creativecommons.org/licenses/by-nc-sa/3.0/)
%
%%%%%%%%%%%%%%%%%%%%%%%%%%%%%%%%%%%%%%%%%

%----------------------------------------------------------------------------------------
%	PACKAGES AND THEMES
%----------------------------------------------------------------------------------------

\documentclass{beamer}

\mode<presentation> {

% The Beamer class comes with a number of default slide themes
% which change the colors and layouts of slides. Below this is a list
% of all the themes, uncomment each in turn to see what they look like.

%\usetheme{default}
%\usetheme{AnnArbor}
%\usetheme{Antibes}
%\usetheme{Bergen}
%\usetheme{Berkeley}
%\usetheme{Berlin}
%\usetheme{Boadilla}
%\usetheme{CambridgeUS}
%\usetheme{Copenhagen}
%\usetheme{Darmstadt}
%\usetheme{Dresden}
%\usetheme{Frankfurt}
%\usetheme{Goettingen}
%\usetheme{Hannover}
%\usetheme{Ilmenau}
%\usetheme{JuanLesPins}
%\usetheme{Luebeck}
\usetheme{Madrid}
%\usetheme{Malmoe}
%\usetheme{Marburg}
%\usetheme{Montpellier}
%\usetheme{PaloAlto}
%\usetheme{Pittsburgh}
%\usetheme{Rochester}
%\usetheme{Singapore}
%\usetheme{Szeged}
%\usetheme{Warsaw}

% As well as themes, the Beamer class has a number of color themes
% for any slide theme. Uncomment each of these in turn to see how it
% changes the colors of your current slide theme.

%\usecolortheme{albatross}
%\usecolortheme{beaver}
%\usecolortheme{beetle}
%\usecolortheme{crane}
%\usecolortheme{dolphin}
%\usecolortheme{dove}
%\usecolortheme{fly}
%\usecolortheme{lily}
%\usecolortheme{orchid}
%\usecolortheme{rose}
%\usecolortheme{seagull}
%\usecolortheme{seahorse}
%\usecolortheme{whale}
%\usecolortheme{wolverine}

%\setbeamertemplate{footline} % To remove the footer line in all slides uncomment this line
%\setbeamertemplate{footline}[page number] % To replace the footer line in all slides with a simple slide count uncomment this line

%\setbeamertemplate{navigation symbols}{} % To remove the navigation symbols from the bottom of all slides uncomment this line
}

\usepackage{graphicx} % Allows including images
\usepackage{booktabs} % Allows the use of \toprule, \midrule and \bottomrule in tables

\usepackage{amsmath}
\usepackage{bm}
\usepackage{hyperref}

% \usepackage{algorithm}
\usepackage{algorithm,caption}
\usepackage[options ]{algorithm2e}
\usepackage[noend]{algpseudocode}

\newcommand{\dataset}{{\cal D}}
\newcommand{\fracpartial}[2]{\frac{\partial #1}{\partial  #2}}
\newcommand{\tr}{\intercal}
\newcommand{\eye}{\mathrm{I}}
\newcommand\given[1][]{\:#1\vert\:}
\newcommand{\transpose}[1]{#1^{\intercal}}
\newcommand{\R}{\mathbb{R}}
\newcommand{\nprod}{\prod_{n}}
\newcommand{\kprod}{\prod_{k}}
\newcommand{\nsum}{\sum_{n}}
\newcommand{\ksum}{\sum_{k}}
\newcommand{\boldbeta}{\boldsymbol\beta}
\newcommand{\boldgamma}{\boldsymbol\gamma}
\newcommand{\boldtau}{\boldsymbol\tau}
\newcommand{\sumexp}{\sum_{j=1}^{K} \exp \{ \transpose{x_n} \gamma_j \}}
\newcommand{\E}{\mathbb{E}}


\usepackage{setspace}
\let\Algorithm\algorithm
\renewcommand\algorithm[1][]{\Algorithm[#1]\setstretch{0.75}}


\newcommand{\pr}[1]{p \left( #1 \right)}



%----------------------------------------------------------------------------------------
%	TITLE PAGE
%----------------------------------------------------------------------------------------

\title[]{Variational Approach for Bayesian Density Regression} % The short title appears at the bottom of every slide, the full title is only on the title page

\author{Eric Chuu} % Your name
\institute[TAMU] % Your institution as it will appear on the bottom of every slide, may be shorthand to save space
{
Texas A\&M University \\ % Your institution for the title page
\medskip
\textit{ericchuu@tamu.com} % Your email address
}
\date{\today} % Date, can be changed to a custom date

\begin{document}

\begin{frame}
\titlepage % Print the title page as the first slide
\end{frame}

\begin{frame}
\frametitle{Overview} % Table of contents slide, comment this block out to remove it
\tableofcontents % Throughout your presentation, if you choose to use \section{} and \subsection{} commands, these will automatically be printed on this slide as an overview of your presentation
\end{frame}

%----------------------------------------------------------------------------------------
%	PRESENTATION SLIDES
%----------------------------------------------------------------------------------------

%------------------------------------------------
\section{Introduction and Setup} % Sections can be created in order to organize your presentation into discrete blocks, all sections and subsections are automatically printed in the table of contents as an overview of the talk
%------------------------------------------------

\section{Main Problem} 

\subsection{Computational Bottlenecks}

% \section{Variational Approach}



%------------------------------------------------

\begin{frame}
\frametitle{Problem Introduction and Setup}
\begin{itemize}
\item Lorem ipsum dolor sit amet, consectetur adipiscing elit
\item Aliquam blandit faucibus nisi, sit amet dapibus enim tempus eu
\item Nulla commodo, erat quis gravida posuere, elit lacus lobortis est, quis porttitor odio mauris at libero
\item Nam cursus est eget velit posuere pellentesque
\item Vestibulum faucibus velit a augue condimentum quis convallis nulla gravida
\end{itemize}
\end{frame}

%------------------------------------------------

\begin{frame}
\frametitle{Main Problem}
\begin{block}{Block 1}
Lorem ipsum dolor sit amet, consectetur adipiscing elit. Integer lectus nisl, ultricies in feugiat rutrum, porttitor sit amet augue. Aliquam ut tortor mauris. Sed volutpat ante purus, quis accumsan dolor.
\end{block}

\begin{block}{Block 2}
Pellentesque sed tellus purus. Class aptent taciti sociosqu ad litora torquent per conubia nostra, per inceptos himenaeos. Vestibulum quis magna at risus dictum tempor eu vitae velit.
\end{block}

\begin{block}{Block 3}
Suspendisse tincidunt sagittis gravida. Curabitur condimentum, enim sed venenatis rutrum, ipsum neque consectetur orci, sed blandit justo nisi ac lacus.
\end{block}
\end{frame}

%------------------------------------------------

\begin{frame}
\frametitle{Multiple Columns}
\begin{columns}[c] % The "c" option specifies centered vertical alignment while the "t" option is used for top vertical alignment

\column{.45\textwidth} % Left column and width
\textbf{Heading}
\begin{enumerate}
\item Statement
\item Explanation
\item Example
\end{enumerate}

\column{.5\textwidth} % Right column and width
Lorem ipsum dolor sit amet, consectetur adipiscing elit. Integer lectus nisl, ultricies in feugiat rutrum, porttitor sit amet augue. Aliquam ut tortor mauris. Sed volutpat ante purus, quis accumsan dolor.

\end{columns}
\end{frame}

%------------------------------------------------
\section{Variational Approach}
%------------------------------------------------

\begin{frame}
\frametitle{Variational Approach}
\begin{table}
\begin{tabular}{l l l}
\toprule
\textbf{Treatments} & \textbf{Response 1} & \textbf{Response 2}\\
\midrule
Treatment 1 & 0.0003262 & 0.562 \\
Treatment 2 & 0.0015681 & 0.910 \\
Treatment 3 & 0.0009271 & 0.296 \\
\bottomrule
\end{tabular}
\caption{Table caption}
\end{table}
\end{frame}

%------------------------------------------------

\begin{frame}
\frametitle{Theorem}
\begin{theorem}[Mass--energy equivalence]
$E = mc^2$
\end{theorem}
\end{frame}

%------------------------------------------------
\subsection{CAVI Updates}
%------------------------------------------------

\begin{frame}
\frametitle{CAVI Updates}
\begin{theorem}[Mass--energy equivalence]
$E = mc^2$
\end{theorem}
\end{frame}

%------------------------------------------------

\begin{frame}[fragile] % Need to use the fragile option when verbatim is used in the slide
\frametitle{Verbatim}
\begin{example}[Theorem Slide Code]
\begin{verbatim}
\begin{frame}
\frametitle{Theorem}
\begin{theorem}[Mass--energy equivalence]
$E = mc^2$
\end{theorem}
\end{frame}\end{verbatim}
\end{example}
\end{frame}

%------------------------------------------------

\begin{frame}
\frametitle{Figure}
Uncomment the code on this slide to include your own image from the same directory as the template .TeX file.
%\begin{figure}
%\includegraphics[width=0.8\linewidth]{test}
%\end{figure}
\end{frame}

%------------------------------------------------


%------------------------------------------------
\subsection{Bouchard Bound for Problematic Quantity}
%------------------------------------------------


\begin{frame}
\frametitle{Bouchard Bound for Problematic Quantity}
\begin{theorem}[Mass--energy equivalence]
$E = mc^2$
\end{theorem}
\end{frame}

%------------------------------------------------

\begin{frame}[fragile] % Need to use the fragile option when verbatim is used in the slide
\frametitle{Citation}
An example of the \verb|\cite| command to cite within the presentation:\\~

This statement requires citation \cite{p1}.
\end{frame}

%------------------------------------------------
\subsection{Variational Algorithm}
%------------------------------------------------


\begin{frame}[fragile] % Need to use the fragile option when verbatim is used in the slide
\frametitle{Variational Algorithm}
\textbf{Input}
\begin{enumerate}
\item Data $\left(y_n, x_n\right)_{n=1}^{N}$
\item Number of components, $K$
\item Prior mean, precision for coefficients vectors, $\boldbeta_{1:K}$
\item Prior shape, rate parameters for precision parameters, $\tau_{1:K}$
\end{enumerate}
\textbf{Output}
\begin{enumerate}
\item A variational density,
$$q \left( \mathbf{Z}, \boldbeta, \boldtau, \boldgamma \right) = q(\mathbf{Z}) q(\boldbeta, \boldtau, \boldgamma) = q(\mathbf{Z}) \kprod q(\beta_k, \tau_k) q(\gamma_k)$$
\item Fully specified by the variational parameters
\end{enumerate}
\end{frame}


\begin{frame}[fragile] % Need to use the fragile option when verbatim is used in the slide
\frametitle{Variational Algorithm}
\begin{algorithm}[H]
\SetAlgoLined
 \While{the ELBO has not converged}{
  
  \For{$n \in \{1, \ldots, N\}$}{
       \For{$k \in \{1, \ldots, K\}$}{
       Set $r_{nk} \propto \exp \Big\{-\frac{1}{2} \ln (2\pi) + \frac{1}{2} \E[\ln \tau_k] + \transpose{x_n} \E[\gamma_k]$ \;
       
       $\qquad \qquad - \frac{1}{2} \E [\tau_k(y_n - \transpose{x_n}\beta_k)^2]\big  - \E\Big[\ln \left( \sumexp \right)\Big]\Big\}$\;
       }
   }

  \For{$k \in \{1, \ldots, K\}$}{ %% order of the following updates matters
      \For{$n \in \{1, \ldots, N\}$}{
          % update xi_1k, ... xi_Nk
	      Set $\xi_{nk} \leftarrow \sqrt{(\transpose{x_n}\mu_k - \alpha_n)^2 + \transpose{x_n}\mathrm{Q}_k^{-1} x_n}$\;          
       }
   }

   $\texttt{ /** Remaining Variational Updates on Next Slide **/ }$
   
}
 \caption*{\textbf{Algorithm}. CAVI for Conditional Density Estimation}
\end{algorithm}
\end{frame}

%------------------------------------------------



\begin{frame}[fragile] % Need to use the fragile option when verbatim is used in the slide
\frametitle{Variational Algorithm (cont.)}
\begin{algorithm}[H]
\SetAlgoLined
 \While{the ELBO has not converged}{
        
      \For{$n \in \{1, \ldots, N\}$}{        
       % update alpha_1,...,alpha_N
       Set $\alpha_{n} \leftarrow \Big[\frac{1}{2} \left( \frac{K}{2} - 1\right) + \ksum \lambda(\xi_{nk}) \transpose{\mu_k} x_n\Big] \Big/ \Big[\sum_{k} \lambda(\xi_{nk})\Big]$\;  
   }
   
   
   \For{$k \in \{1, \ldots, K\}$}{ %% order of the following updates matters
      
   $\mathrm{Q}_{k} \leftarrow \eye_D + 2 \sum_{n} r_{nk} \lambda(\xi_{nk}) x_n \transpose{x_n} \quad \qquad \qquad \quad \texttt{/* gamma\char`_k cov */ }$\; % update Q_k
   
   $\eta_k \leftarrow \sum_{n} r_{nk} \big[ \frac{1}{2} + 2\lambda(\xi_{nk}) \alpha_n \big]x_n$\; % update eta_k
         
   $\mu_k \leftarrow Q_k^{-1} \eta_k \qquad \qquad \qquad \qquad \qquad \qquad \quad \texttt{ /* gamma\char`_k mean */ }$\; % update mu_k = Q_k^{-1} eta_k
   
   $\mathrm{V}_k \leftarrow \nsum r_{nk} x_n \transpose{x_n} + \Lambda_0 \qquad \qquad \quad \qquad \qquad \texttt{ /* beta\char`_k cov */ }$\; % update V_k (update parameters of beta don't directly invvolve tau_k)
   
   $\zeta_k \leftarrow \sum_{n} r_{nk} y_n x_n + \Lambda_0 m_0$\; % update mean_of_beta_k (update parameters of beta don't directly invvolve tau_k)
   
   $m_k \leftarrow \mathrm{V}_k^{-1} \zeta_k \qquad \qquad\qquad\qquad\qquad \qquad \quad \texttt{ /* beta\char`_k mean */ }$\; % update m_k (update parameters of beta don't directly invvolve tau_k)
   
   $a_k \leftarrow a_0 + N_k \qquad \qquad \qquad \qquad \qquad\qquad \quad \texttt{ /* tau\char`_k shape */ }$\; % update a_k
   
   $b_k \leftarrow b_0 + \frac{1}{2}[\nsum r_{nk} y_n^2 + \transpose{m_0}\Lambda_0 m_0 - \transpose{\zeta_k} \mathrm{V}_k^{-1} \zeta_k] \texttt{ /* tau\char`_k rate */ }$\; % update b_k --> this one last since it involves V_k, and mean of beta_k
   }
   Compute ELBO using updated parameters
 } % end of while
 \Return $q \left( \mathbf{Z}, \boldbeta, \boldtau, \boldgamma \right)$
 \caption*{\textbf{Algorithm}. CAVI for Conditional Density Estimation}
\end{algorithm}
\end{frame}



\begin{frame}
\frametitle{References}
\footnotesize{
\begin{thebibliography}{99} % Beamer does not support BibTeX so references must be inserted manually as below
\bibitem[Smith, 2012]{p1} John Smith (2012)
\newblock Title of the publication
\newblock \emph{Journal Name} 12(3), 45 -- 678.
\end{thebibliography}
}
\end{frame}

%------------------------------------------------

\begin{frame}
\Huge{\centerline{Thank you!}}
\end{frame}

%----------------------------------------------------------------------------------------

\end{document}